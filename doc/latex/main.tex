\documentclass{scrreprt}
\usepackage{graphicx} % Required for inserting images
\usepackage{amsmath}

\title{Chemical Reactor}
\author{Leo Basov}
\date{December 2024}

\begin{document}

\maketitle

\chapter{Introduction}
The total energy of the system is given as
\begin{equation}
	E_{tot} = E_{kin} + E_{rot} + E_{vib}
\end{equation}
the individual components can be expressed as assuming a common kinetic temperature of the mixture
\begin{align}
	E_{kin} &= \frac{3}{2} k_B T \sum_i N_i\\
	E_{rot} &= k_B \sum_i^{N_s} N_i \frac{\xi_i}{2} T_i\\
	E_{vib} &= k_B \sum_i^{N_s} \left( N_i \sum_m g_{i, m} \frac{\theta_{i, m}}{\exp(\theta_{i, m} / T_{i, m}) - 1} \right)
\end{align}
where the index $i$ denotes the species and $m$ the vibrational mode.
We assume a constant volume thus we can work with per volume quantities and replace the particle number $N$ with number density $n$.
Assuming a thermalized rotational temperature we can write the per particle energy as
\begin{align}
	e_{kr, i} &= \frac{3 + \xi_i}{2} k_B T\\
	e_{v, i, m} &= k_B g_{m, i} \frac{\theta_{i, m}}{\exp(\theta_{i, m} / T_{i, m}) - 1}
\end{align}
and therefore
\begin{align}
	E_{kin} + E_{rot} &= \sum_i n_i e_{kr, i}\\
	E_{vib} &= \sum_i \left( n_i \sum_m e_{v, i, m}\right)
\end{align}
The change of the total energy in the system is then assuming that the total energy of the system can change only due to chemical reactions is
\begin{align}
	\frac{\partial E_{kin} + E_{rot}}{\partial t} &= -\sum_i \sum_m \frac{\partial E_{vib, i, m}}{\partial t} + \Delta E_R\\
	\frac{\partial E_{vib, i, m}}{\partial t} &= \frac{\partial n_i}{\partial t} e_{v, i, m} + n_i  \frac{\partial e_{v, i, m}}{\partial t}\\
	\Delta E_R &= \sum_r \nu_r \Delta e_r\\
	\frac{\partial n_i}{\partial t} &= \sum_k n_i n_k r_{i, k}(T)\\
	\frac{\partial e_{v, i, m}}{\partial t} &= \frac{e_{v, i, m}(T) - e_{v, i, m}(T_{i,m})}{\tau_{i, m}}.
\end{align}

\end{document}